\documentclass[onecolumn,11pt,a4paper]{article}

\usepackage[utf8]{inputenc}
\usepackage{lmodern}
\usepackage{cmap}                  % fix search and cut-and-paste in Acrobat
\usepackage[T1]{fontenc}
\setlength{\parindent}{0pt}

\usepackage[plainpages=false,pdfcenterwindow=true,
    pdftoolbar=false,pdfmenubar=false,
    pdftitle={},
    pdfauthor={},
    linkcolor=black,citecolor=black,filecolor=black,urlcolor=black]{hyperref}

\usepackage[margin=1.20in,top=1.35in,bottom=1.35in]{geometry}
\usepackage{enumerate}
\usepackage{amsmath, amsfonts, amsthm, amssymb, amsbsy, bbm, latexsym}
\allowdisplaybreaks
\usepackage[numbers]{natbib}
\bibliographystyle{unsrtnat}
\usepackage{tikz}
\usepackage{todonotes}
\usetikzlibrary{arrows,positioning,shapes}

\usepackage{color}                 % use colors with, e.g., \textcolor{red}{text}
\definecolor{darkgray}{rgb}{0.15,0.15,0.15}   % rgb(38, 38, 38)
\definecolor{lightgray}{rgb}{0.94,0.94,0.94}  % rgb(239, 239, 239)
\definecolor{lightlightgray}{rgb}{0.97,0.97,0.97}  % rgb(247, 247, 247)
\definecolor{darkred}{rgb}{0.80,0.00,0.00}    % rgb(204, 0, 0)
\definecolor{darkgreen}{rgb}{0.00,0.70,0.00}    % rgb(0,178,0)
\definecolor{darkblue}{rgb}{0.00,0.00,0.70}    % rgb(0,0,178)

\usepackage{graphicx}              % to include figures
\graphicspath{{fig/}}              % directory in which figures are stored
\usepackage[labelfont=bf]{caption}               % add a caption to figures
\usepackage{subcaption}            % and allow multi figures
\usepackage{multicol}              % multi column environment
\usepackage{float}                 % allow more control to float and figures positions
\usepackage{framed}                % \framed command to frame an important part
\usepackage{palatino}              % Use the Palatino font % XXX remove if it is ugly ?
\usepackage{enumitem}              % For \begin{enumerate}, switch back to itemize if it fails
\usepackage{wasysym}               % Symbol font http://stackoverflow.com/a/190321/5889533, for \frownie
\usepackage{marvosym}               % Amother symbol font, for \Heart
\usepackage{lipsum}
% Custom package for the better footnote symbols, http://ctan.org/pkg/footmisc
\usepackage[bottom]{footmisc}
% \usepackage[perpage,para,symbol*]{footmisc}

\theoremstyle{plain}  % Plain style for theorem, defn, lemma, proposition, corollary
\newtheorem{theorem}{Theorem}[section]
% \newtheorem{proof}{Proof}  % Already defined by amsthm
\newtheorem{defn}[theorem]{Definition}
\newtheorem{attempt}[theorem]{Attempt}
\newtheorem{lemma}[theorem]{Lemma}
\newtheorem{proposition}[theorem]{Proposition}
\newtheorem{property}[theorem]{Property}
\newtheorem{properties}[theorem]{Properties}
\newtheorem{corollary}[theorem]{Corollary}
\theoremstyle{remark}  % Remark style for remark, example, examples
\newtheorem{remark}[theorem]{Remark}
\newtheorem{warning}[theorem]{\textcolor{red}{Warning}}
\newtheorem{example}[theorem]{Example}
\newtheorem{examples}[theorem]{Examples}


\definecolor{theorembar}{rgb}{0.00,0.95,0.00}      % Define a new color rgb(0,242,0)
\definecolor{defnbar}{rgb}{0.00,0.00,0.95}         % Define a new color rgb(0,0,242)
\definecolor{propositionbar}{rgb}{0.58,0.00,1.00}  % Define a new color rgb(147,0,255)
\definecolor{corollarybar}{rgb}{0.00,0.95,0.95}    % Define a new color rgb(0,242,242)
\definecolor{lemmabar}{rgb}{1.00,0.00,0.78}        % Define a new color rgb(255,0,198)
\definecolor{warningbar}{rgb}{0.90,0.00,0.00}      % Define a new color rgb(229,0,0)
\definecolor{propertybar}{rgb}{1.00,1.00,0.00}  % Define a new color rgb(255,255,0)
% \definecolor{examplebar}{rgb}{0.00,0.00,0.00}      % Define a new color rgb(0,0,0)
\definecolor{remarkbar}{rgb}{1.00,0.53,0.00}       % Define a new color rgb(255,135,0)

% Tips from http://tex.stackexchange.com/a/205133/97964
\usepackage[nameinlink]{cleveref}

\usepackage{listings}
\lstset{numbers=left, numberstyle=\tiny,frame=single}

\usepackage{lastpage,fancyhdr}     % customize the headers and footers
\pagestyle{fancy}
    \setlength{\headheight}{14pt}
    \renewcommand{\headrulewidth}{0.2pt}
    \renewcommand{\footrulewidth}{0.2pt}
    \lhead{\emph{Modelling Seminar and Projects}}
    \rhead{\emph{2020-2021}}
    \lfoot{Exactness of Parillo relaxations}
    \cfoot{$\thepage/\pageref{LastPage}$}
    \rfoot{Abdellaziz, Ghimire, Laskina, Tanji}

% Change the name if the table of content from ``Contents'' to ``Outline''
\renewcommand{\contentsname}{\hfill{} Outline \hfill{}}  % XXX Put the title centered in the page!
% XXX Change the font size of the title ``Abstract'' for the abstract
\usepackage{abstract}  % http://www.latex-community.org/forum/viewtopic.php?f=4&t=18791
\renewcommand{\abstractnamefont}{\normalfont\Large\bfseries}
% \renewcommand{\abstractname}{Introduction}

\providecommand*{\hr}[1][class-arg]{%
    \hspace*{\fill}\hrulefill\hspace*{\fill}
    \vskip 0.65\baselineskip
}

% Custom commands:
\newcommand*{\defeq}{\stackrel{\text{def}}{=}}

\newcommand{\eqv}{\Leftrightarrow}
\newcommand{\bm}{\boldsymbol}

\newcommand\cqfd{\hfill\(\blacksquare\)}
\newcommand\qedsp{\hfill\(\square\)}

\newcommand{\argmax}[1]{\underset{#1} {\operatorname{arg\,max}}}

%%%%%%%%%%%%%%%%%%%%%%%%%%%%%%%%%
\usepackage{ragged2e}
\usepackage{afterpage}
\newcommand{\blankpage}{}

%%%%%%%%%%%%%%%%%%%%%%%%%%%%%%%%%%%%%%%%%%%%%%%%%%%%%%%%%%%%%%%%%%%%%%%%%%%%%%
\begin{document}
\DeclareGraphicsExtensions{.pdf,.png,.jpg}
\large
% ------------------------------------------------------------------------------
\title{\vspace*{10pt}\Huge{\textbf{Exactness of SOS relaxations in copositive programming}}\\ \vspace*{15pt} \Large{Research Project Report - "Modelling seminar project" course} }
\thispagestyle{plain}
\date{}
\author{}
\maketitle
% ------------------------------------------------------------------------------
\hr{}
% ------------------------------------------------------------------------------
% Author(s)
\renewcommand{\thefootnote}{\fnsymbol{footnote}}
\begin{Large}
\begin{center}
		\href{mailto:Mohamed-Amine.Abdellaziz@grenoble-inp.org}{\textbf{Mohamed Amine~Abdellaziz}}\\
		\textcolor{blue}{\texttt{{mohamed-amine}{.}{abdellaziz}{{@}}{grenoble-inp}{.}{org}}}\\
		\href{mailto:suraj.ghimire@grenoble-inp.org}{\textbf{Suraj~Ghimire}}\\
    	\textcolor{blue}{\texttt{{suraj}{.}{ghimire}{{@}}{grenoble-inp}{.}{org}}}\\
		\href{mailto:laskina.as@phystech.edu}{\textbf{Anna~Laskina}}\\
    	\textcolor{blue}{\texttt{{laskina}{.}{as}{{@}}{phystech}{.}{edu}}}\\
    	\href{mailto:sofiane.tanji@grenoble-inp.org}{\textbf{Sofiane~Tanji}}\\
		\textcolor{blue}{\texttt{{sofiane}{.}{tanji}{{@}}{grenoble-inp}{.}{org}}}\\
\end{center}
\end{Large}
\renewcommand{\thefootnote}{\arabic{footnote}}
\setcounter{footnote}{0}  %% XXX
% ------------------------------------------------------------------------------
\hr{}

% ------------------------------------------------------------------------------
% A small abstract of what is done in the paper
\begin{normalsize}  % XXX ?
\begin{abstract}
    \begin{normalsize}  % XXX ?
		\justify
		For our research project, we studied the Parrilo relaxations
		for certifying copositivy of a given matrix of size $6 \times 6$. In
		this small project report, we first present quickly the framework of Parrilo relaxations
		with its hypotheses and the usual notations, along with a sum-up of the sums of squares literature.
		Our main contribution is to certify the exactness of Parrilo relaxations on unit diagonal copositive matrices
		for $n = 6$. We first present the framework of Parrilo relaxations quickly while recapitulating notations.
		In section 2, we present the computational approach taken to check the exactness of Parrilo relaxations and the results obtained.
		Then, we present the structure of the sums-of-squares decomposition obtained numerically and explain it in section 3.
		In section 4, we prove the 1-Parrilo cone to be a tractable approximation of the unit-diagonal copositive cone for $n = 6$.
    \end{normalsize}  % XXX ?
\end{abstract}
\end{normalsize}  % XXX ?


% ------------------------------------------------------------------------------
\hr{}

%%%%%%%%%%%%%%%%%%%%%%%%%%%%%%%%%%%%%%%%%%%%%%%%%%%%%%%%%%%%%%%%%%%%%%%%%%%%%%
% Table of Contents - One page ?!
\newpage
\normalsize % XXX ?

\begin{large} % XXX ?
    \tableofcontents
\end{large} % XXX ?

\normalsize % XXX ?
% ------------------------------------------------------------------------------
\hr{}  % Horizontal line, like <hr> in HTML
% XXX remove if it is ugly or take too much space

\vfill{}
% ------------------------------------------------------------------------------
% About the report
\begin{center}
    \begin{large}
        % FIXED He was not really my advisor but allright...
        \textbf{Project Advisor:} \href{http://www-ljk.imag.fr/membres/Roland.Hildebrand/}{Roland Hildebrand} (\href{http://dao-ljk.imag.fr/}{DAO team}, \href{http://www-ljk.imag.fr/}{Laboratoire Jean Kuntzmann}) \\
        \textbf{Course:}
        \emph{\href{https://chamilo.grenoble-inp.fr/courses/ENSIMAGWMM9AM10/index.php}{``Modeling seminar and projects''}}, by \href{http://www-ljk.imag.fr/membres/Emmanuel.Maitre/doku.php}{Emmanuel Maître}, in $2020$, \\
        \textbf{Master $2$ program: } \href{https://msiam.imag.fr/doku.php}{Master of Science in Industrial and Applied Mathematics (MSIAM)}
        at \href{https://ensimag.grenoble-inp.fr/}{Grenoble INP - Ensimag}. \\
    \end{large}
\end{center}
% ------------------------------------------------------------------------------
\newpage

\blankpage

% ------------------------------------------------------------------------------
% ------------------------------------------------------------------------------
\paragraph{Outline:} 
% TODO : repeat the outline but with much more details.
Our contribution corresponds to sections 2, 3 and 4.\\
\hr{}
\section{Presentation}
\label{sec:presentation}
Here are detailed the main results on copositive matrices and its applications along with common approximations of the copositive cone.
% ------------------------------------------------------------------------------
\subsection{The copositive cone}
\label{sub:copositive}
% TODO : talk about copositivity
% TODO : Talk about the common approximation C = S + N
% TODO : Talk about convexity structure for n = 6 and the 5 families.
% ------------------------------------------------------------------------------
\subsection{Parrilo cones and sums of squares}
\label{sub:parrilo}
% TODO : Explain what are Parrilo cones first.
% TOOD : Detail the results linked to the copositive cone.
% TODO : Use figures maybe for n = 3
The next sections aim to prove that every matrix $A \in \mathcal{COP}^6$ with unit diagonal is in $\mathcal{K}_6^1$.
% ------------------------------------------------------------------------------
\section{Numerical certificate of exactness}
\label{sec:computation}
This section presents the computational approach taken to have a numerical certificate.
We detail how we obtained the special copositive matrices and how we checked existence of a sums of squares decomposition.
% ------------------------------------------------------------------------------
\subsection{Generation of random instances}
\label{sub:generation}
% ------------------------------------------------------------------------------
\subsection{Certify a matrix is in the Parrilo cone}
\label{sub:parrilo-certificate}
% ------------------------------------------------------------------------------
\subsection{Implementation of the semi-definite program and results}
\label{sub:sdp}
\subsection{Solving the SDP : issues tackled}
\label{sub:solving-issues}
We quickly present the issues faced to solve the SDP.
% TODO : talk about making kkt-solver robust.
% TODO : talk about the random 2d projection clustered.
% ------------------------------------------------------------------------------
\section{Structure of the SOS decomposition}
\label{sec:sos-structure}
In this section, we detail the structure of the SOS decompositions obtained numerically for each family,
discuss about the monomials participating to the decomposition for each family and what this tells us on
the structure of the problem.
% ------------------------------------------------------------------------------
\subsection{Family 1}
\label{sub:fam-1}
% ------------------------------------------------------------------------------
\subsection{Family 2}
\label{sub:fam-2}
% ------------------------------------------------------------------------------
\subsection{Family 3}
\label{sub:fam-3}
% ------------------------------------------------------------------------------
\subsection{Family 4}
\label{sub:fam-4}
% ------------------------------------------------------------------------------
\subsection{Family 5}
\label{sub:fam-5}
% ------------------------------------------------------------------------------
\section{Analytical certificate of exactness}
\label{sec:analytical}
We now present the main contribution of our project, an analytical certificate of exactness of Parrilo relaxations in the case $n = 6$.
% ------------------------------------------------------------------------------
\subsection{SOS decomposition as a function of angles}
\label{sub:sos-angles}
% ------------------------------------------------------------------------------
\subsection{SOS decomposition and the kernels of the special matrices}
\label{sub:sos-kernel}
% ------------------------------------------------------------------------------
\subsection{Certificate of exactness for each family}
\label{sub:final-result}
% ------------------------------------------------------------------------------
\subsection{Questions still not answered}
\label{sub:questions}
% ------------------------------------------------------------------------------
\section{Conclusion}
% ------------------------------------------------------------------------------
\appendix
% ------------------------------------------------------------------------------
\newpage
\normalsize
% ------------------------------------------------------------------------------
\section{Introduction}
\label{sec:introduction}
\todo{Write a propoer introduction, usefulness of copositive matrices, what do
they represent, what are we going to do, the layout of the report}
% ------------------------------------------------------------------------------
\subsection{Copositive matrices}

\todo{fill the blank spaces with comments to create a smooth flow}
\begin{defn}
	A matrix $A \in \mathcal{M}_{n \times n} \left( \mathbb{R} \right)$ is a \textbf{copositive} matrix if
	$\forall x \in \mathbb{R}_+^n, x^T A x \ge 0$. The set of copositive matrices
	\[ \mathcal{COP}^n = \left\{ A \in \mathcal{M}_{n \times n} \mathbb{R} |
	\text{ $A$ is
		copositive}\right\} \]
	 is called the \textbf{copositive cone}.
\end{defn}

Let us define 

\begin{align*}
	\mathcal{S}_+^n & : \text{ the set of positive semi-definite matrices} \\
	\mathcal{N}_n  & : \text{ the set of element-wise non-negative symmetric matrices} \\
	\label{}
\end{align*}

\todo{Put the references}
\begin{theorem}[Dianonda]
	For $n \le 4$ 
	\[
		 \mathcal{S}_+^n + \mathcal{N}_n =  \mathcal{COP}^n
	\]
\end{theorem}

\begin{theorem}[A. Horn]
	For $n \ge 5$
	\[
		 \mathcal{S}_+^n + \mathcal{N}_n \subsetneq \mathcal{COP}^n
	\]
\end{theorem}

\begin{theorem}
	If $A \in \mathcal{COP}^5$ and $A$ has a unit diagonal then $A \in
	\mathcal{K}_5^1$.
\end{theorem}

The copositive cone $\mathcal{COP}^n$ is invariant under the action of the
multiplicative group $\mathbb{R}^n_{++}, A \mapsto D A D$ where $D = diag(d), d
\in \mathbb{R}^n_{++}$. Thus $A \in \mathcal{COP}^n$ with a non unit positive diagonal can be scaled to a copositive matrix with unit diagonal by setting
$d = diag A$ and applying 
\[
	\tilde{A} = diag\left( d^{-\frac{1}{2}} \right) A diag\left( d^{-\frac{1}{2}}
	\right).
\]

Consequently for $n = 5$ a work-around strategy to check if a matrix $A$ is copositive would
be to scale it to have its diagonal a unit vector and then check if it belongs
to $\mathcal{K}_5^1$. A matrix $A$ belongs to the $r$-th Parrilo cone $\mathcal{K}_n^r$ if the
polynomial of degree $2r + 4$ 

\[
	\left( \sum_{k = 1}^n x^r_k  \right)^r \sum_{K, l = 1}^n A_{k,l} x^2_k x^2_l 
\]

can be represented as a SOS of polynomials of degree $2 + r$.

The goal of the project is to check the exactness of the Parrilo relaxations on
unit diagonal copositive matrices in the $n = 6$ case.

\subsection{Sum of Squares}
\label{sos}
\label{sub:definition}
Let $p: \mathbb{R}^n \rightarrow   \mathbb{R} $ be  any polynomial of degree $2d$. If $p(x)
\geq 0 , \forall x \in \mathbb{R}^n $,  the question is whether can the
polynomial be represented as a sum of squares (SOS), i.e. can be decomposed as 
\[  p(x) =  \sum_{n=1}^m q^{2}_j(x)\] 
where $q_j(x)$ is a homogeneous polynomial of degree $d$, $i = 1,\ldots,m$. 

Let $\mathbf{x} = \left\{ x^\alpha \right\}_{ |\alpha| = d }$ the sequence of all
monomials of degree $d$,

\[ x^\alpha = \prod_{i = 1}^n x_i^{\alpha_i} \]

with $\alpha = \left(
	\alpha_i \right)_{i = 1,\ldots,d} \in \mathbb{N}^n$  and $| \alpha | =
	\sum_{i = 1}^n \alpha_i $ is the degree . 
\begin{proposition}
	A polynomial $p(x)$ is a SOS if and only if there exists a PSD matrix $C$ such
	that $p(x) = \mathbf{x}^T C \mathbf{x}$.
\end{proposition}
	
	\begin{proof}
		We have 
		\begin{equation}
			q_j(x) = c_j^T \mathbf{x}  = \sum_{|\alpha|= d} c_{j, \alpha }^T x^{\alpha} 
		\end{equation}
		where $c_j = \left\{ c_{j, \alpha} \right\}_{|\alpha| = d}$ is the vector of
		coefficients of $q_j$. The component $c_j^T \mathbf{x}$ is a scalar so 

		\begin{equation}\left( c_j^T \mathbf{x}  \right)^T = \mathbf{x}^T c_j = c_j^T \mathbf{x}\end{equation}

		and as a result

		\begin{align*}	\sum_{j = 1}^m q_j (x)^2  & = \sum_{j = 1}^m \left( c_j^T \mathbf{x} 
			\right)^2   = \sum_{j = 1}^m c_j^T \mathbf{x} c_j^T \mathbf{x} \\ & = \sum_{j = 1}^m \mathbf{x}^T c_j
			c_j^T \mathbf{x}   =  \mathbf{x}^T \left( \sum_{j = 1}^m c_j c_j^T  \right)\mathbf{x} \\
			& = \mathbf{x}^T C \mathbf{x} & \\
		\end{align*}

		where $C = \sum_{j = 1}^m c_j c_j^T $ is a PSD matrix. 

		On the other hand any semi-definite matrix $C \succeq 0$ can be written as
		the sum $C = \sum_{j = 1}^m c_j c_j^T$ and equivalently $\mathbf{x}^T C
		\mathbf{x}$ is a SOS.
	\end{proof}

	Verifying that $p(x)$ is a SOS is equivalent to solving the SDP 

	\begin{equation}
		\min_{t, C} t : p(x) = \mathbf{x}^T C \mathbf{x} \text{ and } C + tI
		\succeq 0
		\label{}
	\end{equation}

	If we write
	
	\[
		p(x) = \sum_{|\gamma| = 2d} a_\gamma x^\gamma
	\]
	
	then the equality constraint $p(x) = \mathbf{x}^T C \mathbf{x}$ can be divided into
	simpler constraints on the entries of $C$ by comparing the coeficients at the
	powers of $x$ on both sides, i.e.

	\[
		a_\gamma = \sum_{\alpha + \beta = \gamma} = C_{\alpha, \beta}
	\]

	\begin{example}
		We take a polynomial $p$ of degree $2$ 
	
		\begin{equation*}
			p (x) = 9x_1^2 + 4 x_2^2 + 12 x_1 x_2  
		\end{equation*}

that can be written as the square of a homogeneous matrix $q(x) =  3 x_1 + 2 x_2 $

\begin{equation*}
	p(x) = \left( q(x) \right)^2.
\end{equation*}

We shall find the matrix $ C = \begin{pmatrix}c_{11} & c_{12} \\ c_{21} & c_{22}
\end{pmatrix} $ such that 

\begin{equation*}
	p(x) = \mathbf{x}^T C \mathbf{x}
\end{equation*}

We distribute 

\begin{align*}
	\mathbf{x}^T C \mathbf{x} = & \left( x_1, x_2 \right) \begin{pmatrix}c_{11} & c_{12} \\ c_{21} & c_{22}
	\end{pmatrix} \begin{pmatrix}x_{1} \\ x_{2} \end{pmatrix} \\
	& = \left( c_{11} x_1 + c_{21} x_2, c_{12} x_1 + c_{22} x_2 \right)
	\begin{pmatrix}x_{1} \\ x_{2} \end{pmatrix}  \\
	& = c_{11} x_1^2 + \left( c_{21} + c_{12} \right) x_1 x_2 + c_{22} x_2^2
\end{align*}

The component $c_{11}$ corresponds to $x_1$ the first component of both $\mathbf{x}$
and $\mathbf{x}^T$ so we can note $c_{11} = c_{(1, 0),(1, 0)}$. Also $c_{12}$ corresponds to the second component of $\mathbf{x}$ and to the
first component of $\mathbf{x}^T$ so we can write it as $c_{12} = c_{(1,
0),(0,1)}$. Equivalently we get $c_{21} = c_{(0, 1),(1, 0)} $ and $ c_{22} = c_{(0, 1),(0, 1)}$. So 

\[
	\mathbf{x}^T C \mathbf{x} = c_{(1, 0),(1, 0)} x_1^2 + \left( c_{(0, 1),(1, 0)} + c_{(1,
	0),(0,1)} \right) x_1 x_2 + c_{22} = c_{(0, 1),(0, 1)} x_2^2
\]

We get the system

\begin{equation*}
	\left\{ \begin{aligned}
		a_{(2,0)} & = \sum_{\alpha + \beta = (2, 0)} C_{\alpha \beta} = c_{(1, 0),(1,
		0)} = 9\\
		a_{(1,1)} & = \sum_{\alpha + \beta = (1, 1)} C_{\alpha \beta} = c_{(1, 0),(0, 1)}
		+ c_{(0, 1),(1, 0)} = 4\\
		a_{(0,2)} & = \sum_{\alpha + \beta = (0, 2)} C_{\alpha \beta} = c_{(0, 1),(0,
		1)} = 12\\
	\end{aligned} \right.
\end{equation*}

And a solution would be

\begin{equation*}
	C = \begin{pmatrix}9 & 6 \\ 6 & 4 \end{pmatrix} 
\end{equation*}

\end{example}

\subsection{Sum of Squares}

Let p be temporary quadratic function  $p(x,y,z)$
\[ 
p(x,y,z) = x^4 + y^4 + z^4 - 2x^2yz -  - 2xy^2z -  - 2xyz^2 
\]
\[
X= ( x^2, y^2, z^2, yz, xz, xy)^T 
\]

\textbf{please complete this part}


\begin{align}
\max_{t, c} t: C - tI \geq 0 \\
& a_r = \sum_{\alpha + \beta= r} c_{\alpha  \beta} \forall r
\end{align}

If $t \geq 0 $ then p is SOS representable
if $t<0$ then p is not SOS

$A \in S^n $. A is copositive if $X^T A X \geq 0 \forall X \in \mathbb{R}_{+}^n$ 

Question Is A Co-positive?
\[
\text{A copositive } \Leftrightarrow P_A =\sum_{i, j = 1}^n A_{ij}X_i^2 X_j^2  \geq 0 \]
% ------------------------------------------------------------------------------
\section{Semi-definite programming}
\subsection{Generation of instances of the special matrices}
\label{sub:generation}
% ------------------------------------------------------------------------------
\newpage
\nocite{afonin2020extreme}
\bibliography{references.bib}
\end{document}
